\documentclass[../index.tex]{subfiles}

\begin{document}
    % 1.1
    \section{Đặt vấn đề}
    Trong thời đại công nghệ số hiện nay, Internet đã trở thành một phần không
    thể tách rời trong cuộc sống của hầu hết mọi người. Ngoài việc sử dụng Internet
    để tìm kiếm thông tin, giải trí, người dùng còn có nhu cầu tương tác, chia
    sẻ kiến thức và quan điểm với những người khác. Một trong những hình thức tương
    tác xã hội trực tuyến phổ biến là diễn đàn trực tuyến.

    Diễn đàn trực tuyến là một nền tảng trực tuyến cho phép người dùng tham gia vào
    các cuộc thảo luận, chia sẻ thông tin và trao đổi kiến thức ý kiến về các
    chủ đề cụ thể. Đây là một không gian mở, linh hoạt, cho phép người dùng tham
    gia, xây dựng đóng góp nội dung và tương tác với nhau. Các diễn đàn trực tuyến
    có thể được xây dựng xung quanh các chủ đề như công nghệ, giải trí, thể thao,
    đời sống, sức khỏe, giáo dục, v.v. Sự phát triển của các nền tảng mạng xã
    hội như Facebook, Twitter, Reddit, v.v. đã góp phần thúc đẩy sự quan tâm và
    nhu cầu sử dụng các diễn đàn trực tuyến. Người dùng ngày càng muốn có những không
    gian tương tác trực tuyến, nơi họ có thể chia sẻ kiến thức, trao đổi ý kiến và
    kết nối với những người cùng quan tâm. Các diễn đàn trực tuyến cung cấp một
    môi trường thuận lợi để đáp ứng nhu cầu này.

    Tuy nhiên, xây dựng một diễn đàn trực tuyến hiệu quả và thành công không phải
    là một nhiệm vụ đơn giản. Các vấn đề như thiết kế giao diện thân thiện dễ
    dàng sử dụng, tích hợp các tính năng cần thiết, đảm bảo an toàn và bảo mật
    thông tin, cũng như quản lý, duy trì hoạt động và phát triển của diễn đàn là
    những thách thức cần được giải quyết.

    Xây dựng một diễn đàn trực tuyến là một ý tưởng đáng được thực hiện vì nó
    đáp ứng nhu cầu tương tác xã hội ngày càng tăng của mọi người. Diễn đàn tạo không
    gian để chia sẻ quan điểm, kiến thức và tìm kiếm sự hỗ trợ. Nó thúc đẩy việc
    chia sẻ thông tin, phát triển tri thức cộng đồng và tăng cường kết nối, tương
    tác giữa những người cùng sở thích. Diễn đàn cũng mang lại cơ hội mới cho
    cộng đồng và doanh nghiệp, giúp họ tiếp cận khách hàng, thu thập phản hồi và
    xây dựng cộng đồng. Hơn nữa, diễn đàn có thể tăng khả năng tiếp cận và phát
    triển thương hiệu, đồng thời cung cấp nền tảng tương tác linh hoạt.

    % 1.2
    \section{Mục tiêu và phạm vi đề tài}
    Về giải pháp kỹ thuật, diễn đàn trực tuyến cần được xây dựng trên nền tảng
    website có khả năng mở rộng để đáp ứng số lượng người dùng ngày càng tăng. Giao
    diện người dùng phải thân thiện, dễ sử dụng. Hệ thống cần được tối ưu hóa về
    hiệu suất và tốc độ tải trang, đồng thời đảm bảo bảo mật dữ liệu và thông
    tin cá nhân của người dùng.

    Giải pháp quản lý nội dung đòi hỏi việc xây dựng hệ thống phân loại và tổ
    chức chủ đề một cách rõ ràng, khoa học. Diễn đàn cần có quy trình kiểm duyệt
    nội dung vừa tự động vừa thủ công, kết hợp với công cụ báo cáo vi phạm. Để
    khuyến khích việc tạo nội dung chất lượng, cần thiết lập hệ thống điểm thưởng
    và ghi nhận đóng góp của thành viên. Quan trọng không kém là việc thường
    xuyên lưu trữ và sao lưu dữ liệu để đảm bảo an toàn thông tin.

    Về mặt tương tác cộng đồng, diễn đàn cần tích hợp hệ thống thông báo thời gian
    thực để người dùng không bỏ lỡ các hoạt động quan trọng. Việc cung cấp công
    cụ trò chuyện trực tiếp giữa các thành viên và tính năng theo dõi chủ đề, người
    dùng sẽ tăng cường kết nối trong cộng đồng. Bên cạnh đó, hệ thống đánh giá,
    xếp hạng và tổ chức các sự kiện trực tuyến sẽ tạo động lực cho thành viên
    tham gia tích cực.

    Giải pháp quản trị tập trung vào việc thiết lập hệ thống phân quyền chi tiết
    và xây dựng bảng điều khiển cho quản trị viên. Hệ thống cần có khả năng theo
    dõi và phân tích số liệu hoạt động, quản lý tài khoản và xử lý khiếu nại một
    cách hiệu quả. Việc tự động hóa các quy trình quản lý thường xuyên sẽ giúp tiết
    kiệm thời gian và nguồn lực cho đội ngũ quản trị.

    Để phát triển cộng đồng, diễn đàn cần thường xuyên cập nhật và cải tiến tính
    năng dựa trên phản hồi của cộng đồng sẽ giúp diễn đàn ngày càng hoàn thiện.

    % 1.3
    \section{Định hướng giải pháp}
    Từ những ngày đầu của lập trình, các ứng dụng phần mềm thường được xây dựng như
    một khối thống nhất. Kiến trúc này được gọi là Monolithic, nó đã từng rất phổ
    biến.

    Kiến trúc Monolithic (hay còn gọi là kiến trúc một khối) là một mô hình thiết
    kế phần mềm truyền thống, trong đó toàn bộ ứng dụng được xây dựng và triển
    khai như một đơn vị duy nhất. Tưởng tượng một tòa nhà lớn, tất cả các phòng,
    các hệ thống điện nước, đường ống đều được kết nối và hoạt động cùng nhau
    như một thể thống nhất, đó chính là kiến trúc Monolithic.

    Tất cả các thành phần của ứng dụng (giao diện người dùng, logic nghiệp vụ,
    cơ sở dữ liệu) đều được tích hợp chặt chẽ vào một khối duy nhất. Việc xây
    dựng và triển khai một ứng dụng Monolithic thường đơn giản hơn so với các
    kiến trúc phức tạp khác.

    Tuy nhiên, khi ứng dụng trở nên lớn và phức tạp theo thời gian, việc mở rộng
    và nâng cấp trở nên khó khăn. Việc thay đổi một phần nhỏ trong ứng dụng có thể
    yêu cầu triển khai lại toàn bộ ứng dụng. Với một khối code khổng lồ, việc tìm
    và sửa lỗi trở nên khó khăn hơn khi ứng dụng phát triển. Việc thay đổi công
    nghệ hoặc framework đòi hỏi phải thay đổi toàn bộ ứng dụng.

    Một giải pháp tiềm năng để giải quyết những thách thức trên là Microservices.
    Thay vì xây dựng một ứng dụng lớn, chúng ta chia nhỏ hệ thống thành các dịch
    vụ nhỏ, độc lập. Mỗi dịch vụ sẽ chịu trách nhiệm cho một chức năng cụ thể, như
    quản lý người dùng, lưu trữ bài viết, gửi thông báo,...

    \begin{table}[H]
        \centering
        {}
        \begin{tabular}{ |c|c|c| }
            \hline
            Tính năng  & Kiến trúc Monolithic & Kiến trúc Microservices \\
            \hline
            Mở rộng    & Khó                  & Dễ                      \\
            \hline
            Triển khai & Khó                  & Dễ                      \\
            \hline
            Bảo trì    & Khó                  & Dễ                      \\
            \hline
            Công nghệ  & Đồng nhất            & Đa dạng                 \\
            \hline
            Đội ngũ    & Trung tâm hóa        & Phân tán                \\
            \hline
        \end{tabular}
        \caption{So sánh kiến trúc Monolithic và Microservices}
    \end{table}

    Với kiến trúc Microservices, chúng ta có thể xây dựng một hệ thống diễn đàn
    linh hoạt, dễ mở rộng và bảo trì, đáp ứng được nhu cầu của một cộng đồng lớn.
    Mỗi dịch vụ nhỏ sẽ được phát triển và quản lý bởi một đội ngũ độc lập, giúp
    tăng tốc độ phát triển và giảm thiểu rủi ro.

    Một số hệ thống nổi tiếng sử dụng kiến trúc microservices:
    \begin{itemize}
        \item Netflix: Netflix đã chuyển đổi từ một ứng dụng đơn lẻ sang một hệ thống microservices để đáp ứng nhu cầu ngày càng tăng của người dùng. Điều này giúp Netflix mở rộng quy mô và tăng khả năng phục hồi.
        \item Uber: Uber sử dụng microservices để quản lý các tính năng như đặt xe, thanh toán, định vị và xếp lịch. Điều này cho phép Uber phát triển và triển khai các tính năng mới nhanh chóng.
        \item Amazon: Amazon đã chuyển đổi từ một ứng dụng đơn lẻ sang một kiến trúc microservices để có thể mở rộng quy mô và đáp ứng nhu cầu ngày càng tăng của người dùng.
        \item eBay: eBay đã chuyển đổi sang kiến trúc microservices để tăng tính linh hoạt, khả năng mở rộng và sự độc lập giữa các dịch vụ.
    \end{itemize}

    \newpage
    
    % 1.4
    \section{Bố cục đồ án}
    Phần còn lại của báo cáo đồ án tốt nghiệp gồm các chương chính như sau:

    \textbf{Chương 2: Khảo sát và phân tích nghiệp vụ}
    \newline
    \indent Chương 2 sẽ phân tích yêu cầu, nghiệp vụ của hệ thống, từ đó thiết kế các tác
    nhân, sơ đồ use case và đặc tả cho từng usecase.

    \textbf{Chương 3: Công nghệ sử dụng}
    \newline
    \indent Chương 3 sẽ giới thiệu các công nghệ đã được áp dụng vào hệ thống, bao gồm
    phân tích các ưu điểm và vai trò của công nghệ đó trong hệ thống.

    \textbf{Chương 4: Thiết kế hệ thống}
    \newline
    \indent Chương 4 sẽ phân tích và thiết kế chi tiết hệ thống, bao gồm các thành phần, kiến trúc, áp dụng các công nghệ vào bài toán.

    \textbf{Chương 5: Xây dựng hệ thống}
    \newline
    \indent Chương 5 sẽ tập trung vào việc xây dựng hệ thống.

    \textbf{Chương 6: Các giải pháp và đóng góp nổi bật}
    \newline
    \indent Trong chương này, em sẽ trình bày các giải pháp và đóng góp nổi bật của em trong toàn bộ quá trình làm đồ án.

    \textbf{Chương 7: Kết luận và hướng phát triển}
    \newline
    \indent Chương cuối cùng sẽ tổng kết lại toàn bộ nội dung của đồ án, đánh giá kết quả đạt được, và đề xuất các hướng phát triển, cải tiến trong tương lai.
    
\end{document}